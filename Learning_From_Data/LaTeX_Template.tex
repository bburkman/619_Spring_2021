\documentclass[11pt]{article}
\usepackage{tikz}
\usetikzlibrary{arrows}
\usetikzlibrary{shapes}
\usepackage{pgfmath}
\usepackage{setspace}
\usepackage{amsmath}
\usepackage{array}
\usepackage{hyperref}
\usepackage{enumerate}
\usepackage{enumitem}
\setlist{noitemsep}
\usepackage{listings}
\lstset{language=python}
\usepackage{makeidx}
\usepackage{verbatim}
\usepackage{datetime}

\setlength{\pdfpageheight}{11in}
\setlength{\textheight}{9in}
\setlength{\voffset}{-1in}
\setlength{\oddsidemargin}{0pt}
\setlength{\marginparsep}{0pt}
\setlength{\marginparwidth}{0pt}
\setlength{\marginparpush}{0pt}
\setlength{\textwidth}{6.5in}

\pagestyle{plain}
\makeindex

\title{}
\author{Brad Burkman}
\newdateformat{vardate}{\THEDAY\ \monthname[\THEMONTH]\ \THEYEAR}
\vardate
\date{\today}

\begin{document}
\setlength{\parindent}{20pt}
\begin{spacing}{1.2}
\maketitle
\tableofcontents




%%%%%%%%%%%%%%%%%%
% Index
\clearpage
\addcontentsline{toc}{section}{Index}
\printindex

%%%%%%%%%%%%%%%%
\end{spacing}
\end{document}

%%%%%%%%%%%%
% Useful tools
%%%%%%%%%

\begin{lstlisting}
Put your code here.
\end{lstlisting}

\lstinputlisting[language=python]{source_filename.py}


